\documentclass{beamer}

\mode<presentation> {
  \usetheme{PaloAlto}
}

%%
\makeatletter
\setbeamertemplate{subsubsection in sidebar}{\vspace*{-\baselineskip}}
\setbeamertemplate{subsubsection in sidebar shaded}{\vspace*{-\baselineskip}}
\makeatother
%%

%%
\setbeamertemplate{theorems}[numbered]
%%

\definecolor{Garnet}{RGB}{130,0,20}
\usecolortheme[named=Garnet]{structure}

\logo{\includegraphics[width=1.5cm]{../../sharedImgs/USClogo.png}}

%\setbeamercolor{title}{fg=red!60!black,bg=white!50!black}
%\usecolortheme{beaver}
%\usecolortheme{crane}
\usefonttheme{structuresmallcapsserif}
\usefonttheme[onlysmall]{structurebold}

\usepackage{multicol}
\usepackage{graphicx}
\usepackage{mathtools}
\usepackage{latexsym}
\usepackage{amsfonts}
\usepackage[only,ninrm,elvrm,twlrm,sixrm,egtrm,tenrm]{rawfonts}
\usepackage{indentfirst}
\usepackage[noend]{algorithmic}
\usepackage{algorithm}
\usepackage{enumerate}
\usepackage{graphicx,psfrag}
\usepackage{epsfig}
%\usepackage[pdflatex]{graphicx}
%\usepackage{epstopdf}
\usepackage{ulem}
\usepackage{animate} %need the animate.sty file
\usepackage{tikz}
\usetikzlibrary{fit,shapes,calc}
\usepackage{pgfplots}
\usepackage{amsmath,amsthm,amssymb,amsfonts,enumerate,mymath,mathtools,tikz-cd,mathrsfs}

\newtheorem{thm}{Theorem}
\newtheorem{lem}{Lemma}
\newtheorem{prop}{Proposition}
\theoremstyle{definition}
\newtheorem{defn}{Definition}
\newtheorem{rmk}{Remark}

\newcommand{\A}{\mathscr{A}}
\renewcommand{\C}{\mathscr{C}}

\newcommand*{\defeq}{\mathrel{\vcenter{\baselineskip0.5ex \lineskiplimit0pt
                     \hbox{\scriptsize.}\hbox{\scriptsize.}}}%
                     =}
\DeclarePairedDelimiter\ceil{\lceil}{\rceil}
\DeclarePairedDelimiter\floor{\lfloor}{\rfloor}

\input epsf



\usepackage[english]{babel}
% or whatever

\usepackage[latin1]{inputenc}
% or whatever

\usepackage{times}
\usepackage[T1]{fontenc}
% Or whatever. Note that the encoding and the font should match. If T1
% does not look nice, try deleting the line with the fontenc.

\title % (optional, use only with long paper titles)
    {Rates of Change and Applications to Economics}


\author[Farman]
{Blake Farman~\inst{1}}

\institute[USC]{
\inst{1}
University of South Carolina, Columbia, SC USA}
%\inst{2}
%East Carolina University, Greenville, NC USA\\
%\inst{3}
%University of Johannesburg, Auckland Park, South Africa}

\date[January 12, 2017]
{Math 122: Calculus for Business Administration and Social Sciences}

%\subject{Irredundant and Mixed Ramsey Numbers}
\setbeamercolor{alerted text}{fg=red!60!black}
\setbeamercolor{block title}{bg=white!50!black,fg=red!60!black}

\begin{document}

\begin{frame}
  \titlepage
\end{frame}

\begin{frame}
  \frametitle{Outline}
  \tableofcontents[pausesections]
\end{frame}

\section{1.3: Average Rate of Change and Relative Change}

\begin{frame}{Average Rate of Change}
  \begin{defn}
    The {\it average rate of change} of a function $f$ on an interval $[a,b]$ is
    $$\frac{f(b) - f(a)}{b - a} = \frac{f(a) - f(b)}{a - b}.$$
  \end{defn}
  \onslide<2->{
  \begin{rmk}
    This is just the difference quotient from the last section.
  \end{rmk}}
\end{frame}

\begin{frame}{Example}
  From Columbia, it's about 104 miles to Charleston.
  \onslide<2->{If you make the drive in two hours, what was your average speed?}\\

  
  \onslide<3->{Take Columbia to be distance zero, and mark the starting time at $t = 0$.}
  \onslide<4->{The average speed is:
    $$\onslide<5->{\frac{104 - 0}{2 - 0}} \onslide<6->{= \frac{104}{2}} \onslide<7->{= 52\ \text{mph}}.$$}
  \onslide<8->{
    \begin{rmk}
      Note that this does not necessarily imply you drove 52 mph the entire time, but rather you averaged 52 mph.
    \end{rmk}
    }
\end{frame}

\begin{frame}{Example}
  Find the average rate of change of $f(x) = \sqrt{x}$ on $[1,4]$.

  \onslide<2->{
    $$\frac{f(4) - f(1)}{4 - 1} \onslide<3->{= \frac{\sqrt{4} - \sqrt{1}}{4 - 1}} \onslide<4->{= \frac{2 - 1}{3}} \onslide<5->{= \frac{1}{3}.}$$}
\end{frame}

\begin{frame}{Relative Change}
  Given a quantity, $P$, the {\it relative change} of the quantity from $P$ to $P^\prime$ is
  $$\frac{P^\prime - P}{P}.$$
\end{frame}

\begin{frame}{Example}
  If gas costs $\$2.25$ and the price increases by $\$2$, then find the relative change in price.
  \onslide<2->{
    $$\frac{4.25 - 2.25}{2.25} \onslide<3->{= \frac{2}{2.25}} \onslide<4->{= \frac{2}{\frac{9}{4}}} \onslide<5->{= \frac{8}{9} = 0.\overline{8}}.$$
    }
\end{frame}

\begin{frame}{Example}
  A pair of jeans costs $75.99$ normally.
  \onslide<2->{Today they are on sale for $52.99$.}
  \onslide<3->{What is the relative change in the price?}
  \onslide<4->{
    $$\frac{52.99 - 75.99}{75.99} \onslide<5->{= \frac{-23}{75.99}} \onslide<6->{\approx -0.303.}$$
    }
  \onslide<7->{Hence the price has been reduced by about $30\%$.}
\end{frame}

\begin{frame}{Example}
  The number of sales per week for the jeans above is normally 25.
  \onslide<2->{During the week the jeans are on sale, the number of weekly sales increases to 45.}
  \onslide<3->{Find the relative change in weekly sales.}
  \onslide<4->{
  $$\frac{45 - 25}{25} \onslide<5->{= \frac{20}{25}} \onslide<6->{= \frac{4}{5}.}$$}
  \onslide<7->{Hence weekly sales have increased by $80\%$.}
\end{frame}

\section{1.4: Applications of Functions to Economics}

\begin{frame}{Cost and Revenue}
  Throughout this course we will denote
  \begin{itemize}
    \item<2->
      the cost of producing $q$ goods by $C(q)$,
    \item<3->
      the revenue received from selling $q$ goods by $R(q)$, and
    \item<4->
      the profit from selling $q$ goods by $\pi(q)$.
  \end{itemize}
\end{frame}

\begin{frame}{Example}
  A company makes radios.
  \onslide<2->{
  To begin manufacturing radios, they spend $\$24,000$ on equipment and a factory.}
  \onslide<3->{
  To manufacture a radio costs $\$7$ in material and labour.
  }
  \onslide<4->{
  To manufacture $q$ radios, the cost is:
  $$C(q) = 7q + 24000.$$
  }
  \begin{itemize}
  \item<5->
    The $\$24,000$ expenditue is called a {\it fixed cost}.
  \item<6->
    The $\$7$/radio in labour and material is called a {\it variable cost}.
  \end{itemize}
\end{frame}

\begin{frame}{Linear Marginal Cost}
  \begin{defn}
    For a linear cost function, the marginal cost is the cost to product one additional unit:
    $$\frac{C(q + 1) - C(q)}{(q + 1) - q} = C(q + 1) - C(q).$$
  \end{defn}
  \onslide<2->{
  \begin{rmk}
    This is just the slope of the linear cost function.
  \end{rmk}}
\end{frame}

\begin{frame}{Profit}
  \begin{defn}
    \begin{itemize}
    \item<1->
      Given a revenue and a cost function, the profit function is
      $$\pi(q) = R(q) - C(q).$$
    \item<2->
      The {\it break-even} point is the quantity, $q$, for which
      $$\pi(q) = 0$$
      holds.
    \end{itemize}
  \end{defn}
\end{frame}

\begin{frame}{Example}
  In the example above, assume that radios sell for $15$ each.
  \onslide<2->{The revenue function is
    $$R(q) = 15q.$$}
  \onslide<3->{The profit function is
    $$\pi(q) = R(q) - C(q) \onslide<4->{ = 15q - (7q + 24000)} \onslide<5->{= 8q - 24000.}$$}
  \onslide<6->{The break-even point is value of $q$ making
    $$8q - 24000 = 0$$
    hold.}
  \onslide<7->{Therefore the break-even point is
    $$q = \frac{24000}{8} = 3000.$$}
\end{frame}

\begin{frame}{Marginal Revenue}
  \begin{defn}
    The {\it marginal revenue} for a linear revenue function is the revenue from selling one additional item,
    $$\frac{R(q + 1) - R(q)}{(q + 1) - q} = R(q + 1) - R(q).$$
  \end{defn}
  \onslide<2->{
      \begin{rmk}
        This is just the slope of the revenue function.
      \end{rmk}
    }
\end{frame}
    
\begin{frame}{Marginal Profit}
  \begin{defn}
    The {\it marginal profit} for linear cost and revenue functions is the profit from selling one additional item
    $$\frac{\pi(q+1) - \pi(q)}{(q + 1) - q} = \pi(q + 1) - \pi(q).$$
  \end{defn}
  
  \onslide<2->{
    \begin{rmk}
      This is the slope of the revenue function less the slope of the cost function.
    \end{rmk}
  }
\end{frame}

\end{document}
