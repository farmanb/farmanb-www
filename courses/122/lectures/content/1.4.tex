\documentclass[Lecture.tex]{subfiles}
\begin{document}
\section{1.4: Applications of Functions to Economics}

\begin{frame}{Cost and Revenue}
  Throughout this course we will denote
  \begin{itemize}
    \item<2->
      the cost of producing $q$ goods by $C(q)$,
    \item<3->
      the revenue received from selling $q$ goods by $R(q)$, and
    \item<4->
      the profit from selling $q$ goods by $\pi(q)$.
  \end{itemize}
\end{frame}

\begin{frame}{Example}
  A company makes radios.
  \onslide<2->{
  To begin manufacturing radios, they spend $\$24,000$ on equipment and a factory.}
  \onslide<3->{
  To manufacture a radio costs $\$7$ in material and labour.
  }
  \onslide<4->{
  To manufacture $q$ radios, the cost is:
  $$C(q) = 7q + 24000.$$
  }
  \begin{itemize}
  \item<5->
    The $\$24,000$ expenditue is called a {\it fixed cost}.
  \item<6->
    The $\$7$/radio in labour and material is called a {\it variable cost}.
  \end{itemize}
\end{frame}

\begin{frame}{Linear Marginal Cost}
  \begin{defn}
    For a linear cost function, the marginal cost is the cost to product one additional unit:
    $$\frac{C(q + 1) - C(q)}{(q + 1) - q} = C(q + 1) - C(q).$$
  \end{defn}
  \onslide<2->{
  \begin{rmk}
    This is just the slope of the linear cost function.
  \end{rmk}}
\end{frame}

\begin{frame}{Profit}
  \begin{defn}
    \begin{itemize}
    \item<1->
      Given a revenue and a cost function, the profit function is
      $$\pi(q) = R(q) - C(q).$$
    \item<2->
      The {\it break-even} point is the quantity, $q$, for which
      $$\pi(q) = 0$$
      holds.
    \end{itemize}
  \end{defn}
\end{frame}

\begin{frame}{Example}
  In the example above, assume that radios sell for $15$ each.
  \onslide<2->{The revenue function is
    $$R(q) = 15q.$$}
  \onslide<3->{The profit function is
    $$\pi(q) = R(q) - C(q) \onslide<4->{ = 15q - (7q + 24000)} \onslide<5->{= 8q - 24000.}$$}
  \onslide<6->{The break-even point is value of $q$ making
    $$8q - 24000 = 0$$
    hold.}
  \onslide<7->{Therefore the break-even point is
    $$q = \frac{24000}{8} = 3000.$$}
\end{frame}

\begin{frame}{Marginal Revenue}
  \begin{defn}
    The {\it marginal revenue} for a linear revenue function is the revenue from selling one additional item,
    $$\frac{R(q + 1) - R(q)}{(q + 1) - q} = R(q + 1) - R(q).$$
  \end{defn}
  \onslide<2->{
      \begin{rmk}
        This is just the slope of the revenue function.
      \end{rmk}
    }
\end{frame}
    
\begin{frame}{Marginal Profit}
  \begin{defn}
    The {\it marginal profit} for linear cost and revenue functions is the profit from selling one additional item
    $$\frac{\pi(q+1) - \pi(q)}{(q + 1) - q} = \pi(q + 1) - \pi(q).$$
  \end{defn}
  
  \onslide<2->{
    \begin{rmk}
      This is the slope of the revenue function less the slope of the cost function.
    \end{rmk}
  }
\end{frame}
\end{document}
