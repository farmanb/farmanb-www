\documentclass[Lecture.tex]{subfiles}
\begin{document}
\section{1.6: Logarithms}

\subsection{Inverse Functions}

\begin{frame}{Definition}
  \begin{defn}
    A function $f(x)$ has an {\it inverse} if there exists a function $f^{-1}(x)$ such that
    $$f \circ f^{-1}(x) = x\ \text{and}\ f^{-1} \circ f(x) = x.$$
  \end{defn}
  
  \onslide<2->{
    \begin{thm}[Horizontal Line Test]
      If any horizontal line intersects the graph of $f(x)$ in {\bf at most one} point, then $f(x)$ admits a composition inverse.
    \end{thm}
  }
\end{frame}

\subsection{Definition}

\begin{frame}{Definition}
  First, we note that any exponential function visibly passes the Horizontal Line Test.
  \onslide<2->{
  \begin{defn}
    The {\it logarithm with base a} is the inverse function of the exponential function, $a^x$, and is denoted by
    $$\log_{a}(x).$$
  \end{defn}
  }
  \onslide<3->{
    \begin{rmk}
      \begin{itemize}
      \item<3->
        By definition, 
        $$\log_a(a^x) = x\ \text{and}\ a^{\log_a(x)} = x.$$
      \item<4->
        One denotes $\log_e(x)$ by $\ln(x)$.
      \end{itemize}
    \end{rmk}
    }
\end{frame}

\begin{frame}{Properties of Logarithms}
\begin{itemize}
  \item<1->
    $\log_a(xy) = \log_a(x) + \log_a(y)$,
  \item<2->
    $\log_a\left(\frac{x}{y}\right) = \log_a(x) - \log_a(y)$,
  \item<3->
    $\log_a\left(x^r\right) = r\log_a(x)$,
  \item<4->
    $\log_a(x) = \frac{\log_b(x)}{\log_b(a)}$.
\end{itemize}
\end{frame}

\begin{frame}{Example}
  Solve $3^t = 10$ for $t$.
  \begin{eqnarray*}
    \onslide<2->{\Rightarrow \ln(3^t) &=& \ln(10)\\}
    \onslide<3->{\Rightarrow t\ln(3) &=& \ln(10)\\}
    \onslide<4->{\Rightarrow t &=& \frac{\ln(10)}{\ln(3)} (= \log_3(10))}
  \end{eqnarray*}
\end{frame}

\begin{frame}{Example}
  Solve $12 = 5e^{3t}$ for $t$.
  \begin{eqnarray*}
    \onslide<2->{\Rightarrow e^{3t} &=& \frac{12}{5}\\}
    \onslide<3->{\Rightarrow \ln(e^{3t}) &=& 3t = \ln\left(\frac{12}{5}\right)\\}
    \onslide<4->{\Rightarrow t &=& \frac{1}{3}\ln\left(\frac{12}{5}\right)}
  \end{eqnarray*}
\end{frame}

\subsection{Exponential Functions with Base $e$}

\begin{frame}
  With the natural logarithm, we can rewrite any exponential function with base $e$ if we so choose.
  \onslide<2->{Say, $P(t) = P_0a^t$.}
  \onslide<3->{We let $k = \ln(a)$ so $e^k = a$ and hence 
    $$P_0e^{kt} = P_0\left(e^k\right)^t \onslide<4->{= P_0a^t} \onslide<5->{ = P(t)}$$}
  \onslide<6->{We call $k$ the {\it continuous growth/decay rate}.}
\end{frame}

\begin{frame}{Example}
  Convert $P(t) = 1000e^{0.05t}$ to the form $P_0a^t$.
  
  \onslide<2->{Let $a = e^{0.05}$.}
  \onslide<3->{Then
    $$P(t) = 1000e^{0.05t} = 1000(e^{0.05})^t = 1000a^t.$$}
\end{frame}

\begin{frame}{Example}
  Convert $P(t) = 500(1.06)^t$ to the form $P_0e^{kt}$.
  
  \onslide<2->{$$P(t) = 500(1.06)^t = 500e^{\ln(1.06)t}.$$}
\end{frame}
\end{document}
