\documentclass[Lecture.tex]{subfiles}
\begin{document}
\section{1.7: Exponential Growth and Decay}
\subsection{Doubling Time and Half-Life}
\begin{frame}{Definition}
  \begin{defn}
    \begin{itemize}
    \item<1->
      The {\it doubling time} of an exponentially increasing quantity is the time required for the quantity to double.
    \item<2->
      The {\it half-life} of an exponentially decaying quantity is the time required for the quantity to be reduced by a factor of one half.
    \end{itemize}
  \end{defn}
\end{frame}

\begin{frame}{Doubling Time}
  Every exponentially increasing function, $P(t) = P_0a^t$, has a fixed doubling time, $d$.
  \onslide<2->{Take $d = \log_a(2)$.}
  \onslide<3->{Then
    \begin{eqnarray*}
      \onslide<3->{P(t + d) &=& P_0a^{t + d}\\}
      \onslide<4->{&=& P_0a^ta^d\\}
      \onslide<5->{&=& P_0a^t a^{\log_a(2)}\\}
      \onslide<6->{&=& 2P_0a^t\\}
      \onslide<7->{&=& 2P(t).}
  \end{eqnarray*}}
\end{frame}

\begin{frame}{Half-Life}
  Similarly, every exponentially decreasing function, $P(t) = P_0a^t$, has a fixed half-life, $h$.
  \onslide<2->{Take $$h = \log_a\left(\frac{1}{2}\right) = -\log_a(2).$$}
  \onslide<3->{Then
    \begin{eqnarray*}
      \onslide<3->{P(t + h) &=& P_0a^{t + h}\\}
      \onslide<4->{&=& P_0a^ta^h\\}
      \onslide<5->{&=& P_0a^t a^{-\log_a(2)}\\}
      \onslide<6->{&=& \frac{1}{2}P_0a^t\\}
      \onslide<7->{&=& \frac{1}{2}P(t).}
  \end{eqnarray*}}
\end{frame}

\begin{frame}{Computing Doubling Time/Half-Life}
  To approximate the value of the doubling time with a calculator:
  $$d = log_a(2) = \frac{\ln(2)}{\ln(a)}$$
  and
  $$h = -\log_a(2) = -\frac{\ln(2)}{\ln(a)}.$$
\end{frame}

\begin{frame}{Example}
  Raditiation from an iodine source decays at a continuous hourly rate of $k = -0.004$.
  \onslide<2->{If the radiation level at a spill is about 2.4 millirems/hour:}
  \begin{enumerate}[(a)]
  \item<3->
    What was the radiation level 24 hours later?
  \item<4->
    How long will it take for the radiation levels to decay to the maximum acceptable radiation level of 0.6 millirems/hour set by the EPA?
  \end{enumerate}
\end{frame}

\begin{frame}{Example (Cont.)}
  \begin{enumerate}[(a)]
  \item<1->
    The radiation level 24 hours later is 
    $$R(24) = 2.4 e^{-0.004\cdot 24} \approx 2.18\ \text{millirems/hour}.$$
  \item<2->
    Solve the equation below for $t$:
    \begin{eqnarray*}
      \onslide<3->{0.6 &=& 2.4e^{-0.004t}\\}
      \onslide<4->{\Rightarrow e^{-0.004t} &=& \frac{2.4}{0.6} = \frac{1}{4}\\}
      \onslide<5->{\Rightarrow -0.004t &=& \ln\left(\frac{1}{4}\right) = -\ln(4)\\}
      \onslide<6->{\Rightarrow t &=& \frac{1}{0.004}\ln(4) \approx 346.57\ \text{hours}.\\}
    \end{eqnarray*}
    \onslide<7->{Therefore, it will take approximately $346.57/24 = 14.4$ days.}
  \end{enumerate}
\end{frame}

\begin{frame}
  The population of Kenya was about 19.5 million in 1984 and 39 million in 2009.
  \onslide<2->{Find, assuming exponential growth, a function of $t$ years since 1984 modeling the population.}
  
  \onslide<3->{We are given $P_0 = 19.5$ and $P(25) = 39$.}
  \onslide<4->{If we assume that $P(t) = 19.5e^{kt}$, then
    \begin{eqnarray*}
      \onslide<5->{39 &=& 19.5e^{25k}\\}
      \onslide<6->{\Rightarrow \frac{39}{19.5} &=& 2 = e^{25k}\\}
      \onslide<7->{\Rightarrow \ln(2) &=& \ln(e^{25k}) = 25k\\}
      \onslide<8->{\Rightarrow k &=& \frac{ln(2)}{25}} \onslide<9->{\approx 0.028.}
  \end{eqnarray*}}
  \onslide<10->{Therefore 
    $$P(t) \approx 19.5e^{0.28t}.$$}
\end{frame}

\begin{frame}
  The release of chlorofluorocarbons (CFCs) used in air conditioners and household aerosols destroys the ozone layer in the upper atmosphere.
  \onslide<2->{The quantity of ozone, $Q(t)$, decays exponentially at a continuous rate of $0.25\%$ per year.}
  \onslide<3->{What is the half-life of ozone?\\}

  \onslide<4->{The half life is given by
    \begin{eqnarray*}
      \onslide<4->{\log_{e^k}(2) &=& -\frac{\ln(2)}{\ln(e^k)}\\}
      \onslide<5->{&=& -\frac{ln(2)}{k}\\}
      \onslide<6->{&=& -\frac{ln(2)}{-\frac{1}{400}}\\}
      \onslide<7->{&=& 400\ln(2)} \onslide<8->{\approx 277\ \text{years}.}
  \end{eqnarray*}}
\end{frame}
\subsection{Financial Applications}
\begin{frame}{Compound Interest}
  Assume a sum of money $P_0$ is deposited in an account paying interest at a rate of $r$ yearly, compounded $n$ times per year.
  \onslide<2->{This means that each compounding period, the account earns interest on the balance at a rate of $r/n$.}\\
  
  \onslide<3->{What is the balance of the account after $t$ years?}
\end{frame}

\begin{frame}{Compounding Interest (Cont.)}
  Consider the table:

  \begin{center}
    \begin{tabular}{cc}
      Compounding Period & Account Balance\\
      \onslide<2->{$1$ & $P_0\left(1 + \frac{r}{n}\right)$\\}
      \onslide<3->{$2$ & $P_0\left(1 + \frac{r}{n}\right)\left(1 + \frac{r}{n}\right) = P_0\left(1 + \frac{r}{n}\right)^2$\\}
      \onslide<4->{$3$ & $P_0\left(1 + \frac{r}{n}\right)^2\left(1 + \frac{r}{n}\right) = P_0\left(1 + \frac{r}{n}\right)^3$\\}
      \onslide<5->{\vdots & \vdots \\}
      \onslide<6->{$n$ & $P_0\left(1 + \frac{r}{n}\right)^n$\\}
    \end{tabular}
    \onslide<7->{So at the end of the year, the balance will be $P_0\left(1 + \frac{r}{n}\right)^n$.}
    \onslide<8->{Continuing this way, the account balance after $t$ years will be
      $$P_0\left(1 + \frac{r}{n}\right)^{nt}.$$}
  \end{center}
\end{frame}

\begin{frame}{Doubling Time}
  Say you invest $P_0$ dollars at a rate of $r$ per year, compounded $n$ times.
  \onslide<2->{What is the doubling time?}
  
  
  \onslide<3->{The function for the account balance is
    $$P_0\left(1 + \frac{r}{n}\right)^{nt} = P_0\left(\left(1 + \frac{r}{n}\right)^n\right)^t.$$}
  \onslide<4->{Therefore the doubling time is
    $$d = \log_{\left(1 + \frac{r}{n}\right)^n}(2)\onslide<5->{= \frac{\ln(2)}{\ln\left(\left(1 + \frac{r}{n}\right)^n\right)}}\onslide<6->{ = \frac{ln(2)}{n\ln\left(1 + \frac{r}{n}\right)}.}$$}
\end{frame}

\begin{frame}{Example}
  Say the interest rate is $2\%$ and interest is compounded yearly.
  \onslide<2->{The expected doubling time is
    $$d = \frac{\ln(2)}{\ln(1.02)}\onslide<3->{\approx 35\ \text{years}.}$$}
  
  \onslide<4->{
    \begin{rmk}[``Rule of 70'']
      When $r\%$ is very small,
      $$\ln\left(1 + \frac{r}{100}\right) \approx \frac{r}{100}$$
      and $\ln(2) \approx .7$, so the doubling rate is approximately
      $$\onslide<5->{d = \frac{\ln(2)}{\ln\left(1 + \frac{r}{100}\right)}} \onslide<6->{ \approx \frac{.7}{r/100}}\onslide<7->{ = \frac{70}{r}.}$$
  \end{rmk}}
\end{frame}

\subsection{Continuously Compounding Interest}
\begin{frame}{Continuously Compounding Interest}
  The method above is discrete.
  \onslide<2->{If instead, we wish to compound interest at every instant, we get {\it continuously compounding interest},
    $$P(t) = P_0e^{rt}.$$}
\end{frame}

\begin{frame}{Example}
  If $\$10,000$ is invested at $5\%$ per year, compounded continuously, how long will it take to reach $\$15,000$?
  
  \onslide<2->{We want to solve the equation below for $t$:}
  \begin{eqnarray*}
    \onslide<2->{P(t) &=& 10000e^{t/20} = 15000\\}
    \onslide<3->{\Rightarrow e^{t/20} &=& \frac{15000}{10000} = \frac{3}{2}\\}
    \onslide<4->{\Rightarrow t/20 &=& \ln(e^{t/20}) = \ln\left(\frac{3}{2}\right)\\}
    \onslide<5->{\Rightarrow t &=& 20\ln\left(\frac{3}{2}\right)\\}
    \onslide<6->{&\approx& 8\ \text{years}.}
  \end{eqnarray*}
\end{frame}

\begin{frame}{Doubling Time}
  Say you invest $P_0$ dollars at a rate of $r\%$ per year compounding continuously.
  \onslide<2->{The account balance is given by the function
    $$P_0e^{\frac{r}{100}t} = P_0 (e^{\frac{r}{100}})^t.$$}
  \onslide<3->{Hence the doubling time is given by}
  $$\onslide<3->{\log_{e^{\frac{r}{100}}}(2) = \frac{\ln(2)}{\ln(e^{\frac{r}{100}})}}\onslide<4->{ = \frac{ln(2)}{\frac{r}{100}}}\onslide<5->{ \approx \frac{70}{r}.}$$
\end{frame}
\end{document}
