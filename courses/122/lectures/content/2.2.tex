\documentclass[Lecture.tex]{subfiles}
\begin{document}
\section{2.2: The Derivative Function}
\begin{frame}{Definition}
  \begin{defn}
    \begin{itemize}
    \item<1->
      If a function, $f$, has a derivative at every point in its domain, then we say that $f$ is {\it differentiable}.
    \item<2->
      In this case, we can define a function $f^\prime(x)$ that outputs the instantaneous rate of change of $f$ at $x$.
    \item<3->
      We call $f^\prime(x)$ the {\it derivative function}.
    \end{itemize}
  \end{defn}
\end{frame}

\begin{frame}{The Tangent Line}
  \begin{defn}
    \begin{itemize}
    \item<1->
      We can regard $f^\prime(x_0)$ as a velocity by viewing it as the slope of a line passing through $(x_0,f(x_0))$.
    \item<2->
      We call the line
      $$y - f(x_0) = f^\prime(x_0)(x - x_0)$$
      the {\it line tangent to $f$ at $(x_0, f(x_0))$}.
    \end{itemize}
  \end{defn}
\end{frame}

\begin{frame}{Linearization}
  \begin{itemize}
  \item<1->
    Since we defined $f^\prime(x_0)$ by a limit, 
    $$f^\prime(x_0) \approx \frac{f(x) - f(x_0)}{x - x_0}$$
    for $x$ close to $x_0$.
  \item<2->
    Writing $\Delta x = x - x_0$ we can get a good linear approximation of $f$ close to $x_0$:
    $$f(x) \approx f^\prime(x)\Delta x + f(x_0)$$
    called the {\it Tangent Line Approximation}.
  \item<3->
    This means $f$ locally looks like a line!
  \end{itemize}
\end{frame}

\begin{frame}{Animation}
  \animategraphics[loop,controls,scale=0.3]{12}{zoom/zoom-}{0}{99}
\end{frame}

\begin{frame}{Non-Differentiable Function}
  Consider the absolute value function
  $$\abs{x} = \left\{\begin{array}{ll}x & \text{if}\ 0 \leq x,\\-x & \text{else}\end{array} \right.$$
  at the point $(0,0)$.
  \begin{itemize}
  \item<2->
    For all $x < 0$, 
    $$\frac{\abs{x} - 0}{x - 0} = \frac{-x}{x} = -1.$$
  \item<3->
    For all $0 < x$,
    $$\frac{\abs{x} - 0}{x - 0} = \frac{x}{x} = 1.$$
  \item<4->
    So the derivative at $(0,0)$ is {\bf not} defined: it's -1 if we approach from left to right, and 1 if right to left.
  \end{itemize}
\end{frame}

\begin{frame}
  What does the derivative tells us about the original function?
  On the interval $(a,b)$, if for all $a \leq x \leq b$
  \begin{itemize}
  \item<2->
    $f^\prime(x) \leq 0$, then $f$ is decreasing on $(a,b)$,
  \item<3->
    $0 \leq f^\prime(x)$, then $f$ is increasing on $(a,b)$,
  \item<4->
    $f^\prime(x) = 0$, then $f$ is constant on $(a,b)$.
  \end{itemize}
\end{frame}
\end{document}
