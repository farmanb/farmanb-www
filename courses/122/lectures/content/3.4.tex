\documentclass[Lecture.tex]{subfiles}
\begin{document}
\section{3.4: The Product and Quotient Rules}

\begin{frame}{Product Rule}
  If $f$ and $g$ are differentiable functions, then 
  $$\ddx{x}(f(x)g(x)) = f^\prime(x)g(x) + f(x)g^\prime(x).$$
\end{frame}

\begin{frame}{Quotient Rule}
  Assume that $f$ and $g$ are differentiable functions and $f(x)/g(x)$ is well-defined.
  \onslide<2->{Using the Product and Chain Rules we can compute the derivative of the quotient:}
  \begin{eqnarray*}
    \onslide<3->{\ddx{x}\left(\frac{f(x)}{g(x)}\right)&=& \ddx{x}\left(f(x) g(x)^{-1}\right)\\}
    \onslide<4->{&=& f^\prime(x)g(x)^{-1} + f(x)(-1)g(x)^{-2}g^\prime(x)\\}
    \onslide<5->{&=& \frac{f^\prime(x)}{g(x)} - \frac{f(x)g^\prime(x)}{g(x)^2}\\}
    \onslide<6->{&=& \frac{f^\prime(x)g(x) - f(x)g^\prime(x)}{g(x)^2}.}
  \end{eqnarray*}
\end{frame}

\begin{frame}{Example}
  Differentiate
  \begin{multicols}{3}
    \begin{enumerate}[(a)]
    \item<alert@2-4>
      $x^2e^{2x}$
    \item<alert@5-7>
      $t^3\ln(t+ 1)$
    \item<alert@8->
      $(3x^2 + 5x)e^x$
    \end{enumerate}
  \end{multicols}    
  
  \only<2-4>{
    \begin{eqnarray*}
      \onslide<2-4>{\ddx{x}(x^2e^{2x})}\onslide<3-4>{ &=& 2xe^{2x} + x^2(2e^{2x})\\} \onslide<4>{&=& 2xe^{2x}(1 + x)}
    \end{eqnarray*}
  }
  \only<5-7>{
    \begin{eqnarray*}
      \onslide<5-7>{\ddx{t}(t^3\ln(t+1))}\onslide<6-7>{ &=& 3t^2\ln(t+1) + t^3\left(\frac{1}{t+1}\right)}\\
      \onslide<7>{&=& 3t^2\ln(t+1) + \frac{t^3}{t+1}.}
    \end{eqnarray*}
  }
  \only<8->{
    \begin{eqnarray*}
      \onslide<8->{\ddx{x}(3x^2 + 5x)e^x}\onslide<9->{&=& (6x + 5)e^x + (3x^2 + 5x)e^x\\}
      \onslide<10->{&=&e^x(6x + 5 + 3x^2 + 5x)\\}
      \onslide<11->{&=& e^x(3x^2 + 11x + 5).}
    \end{eqnarray*}
  }
\end{frame}

\begin{frame}{Example}
  Differentiate $\frac{e^{2t}}{t}$.
  \begin{eqnarray*}
    \onslide<2->{\ddx{t}\frac{e^{2t}}{t}} \onslide<3->{ &=& \frac{2e^{2t}t - e^{2t}(1)}{t^2}}\\
    \onslide<4->{&=& \frac{(2t - 1)e^{2t}}{t^2}}
  \end{eqnarray*}
\end{frame}

\begin{frame}{Example}
  A product's price, $p$, is given by
  $$p(q) = 80e^{-0.003q},$$
  where $q$ is the quantity sold.
  \begin{enumerate}[(a)]
    \item<2-|alert@4-6>
      Find the revenue as a function of the quantity sold.
    \item<3-|alert@7->
      How does revenue vary with respect to quantity?
  \end{enumerate}

  \onslide<4->{
    $$\onslide<4->{R(q) = p(q)\cdot q} \onslide<5->{= 80e^{-0.003q}q} \onslide<6->{ = 80qe^{-0.003q}.}$$
    }
  \onslide<7->{
    \begin{eqnarray*}
      \onslide<7->{\ddx{q}R(q) &=& \ddx{q}(80qe^{-0.003q})\\}
      \onslide<8->{&=& 80\ddx{q}qe^{-0.003q}\\}
      \onslide<9->{&=& 80(e^{-0.003q} + q(-0.003)e^{-0.003q})\\}
      \onslide<10->{&=& 80e^{-0.003q}(1 - 0.003q).}
    \end{eqnarray*}
  }  
\end{frame}

\begin{frame}{Example}
  Differentiate
  \begin{multicols}{3}
    \begin{enumerate}[(a)]
    \item<alert@2-5>
      $\displaystyle{\frac{5x^2}{x^3 + 1}}$
    \item<alert@6-8>
      $\displaystyle{\frac{1}{1 + e^x}}$
    \item<alert@9->
      $\displaystyle{\frac{e^x}{x^2}}$.
    \end{enumerate}
  \end{multicols}

  \begin{overprint}
    \only<2-5>{
      \begin{eqnarray*}
        \onslide<2-5>{\ddx{x}\left(\frac{5x^2}{x^3 + 1}\right)}\onslide<3-5>{ &=& \frac{10(x^3 + 1) - 5x^2(3x^2)}{(x^3 + 1)^2}\\}
        \onslide<4-5>{&=& \frac{10x^4 + 10x - 15x^2}{(x^3 + 1)^2}\\}
        \onslide<5>{&=& \frac{-5x^4 + 10}{(x^3 + 1)^2}.}
      \end{eqnarray*}
      }
    \only<6-8>{
      \begin{eqnarray*}
        \onslide<6-8>{\ddx{x}\frac{1}{1 + e^x}} \onslide<7-8>{&=& \frac{0(1 + e^x) - (1)e^x}{(1 + e^x)^2}\\}
        \onslide<8>{&=& \frac{-e^x}{(1 + e^x)^2}.}
      \end{eqnarray*}
      }
    \only<9->{
      \begin{eqnarray*}
        \onslide<9->{\ddx{x}\frac{e^x}{x^2}} \onslide<10->{&=& \frac{e^xx^2 - e^x(2x)}{x^4}\\}
        \onslide<11->{&=& \frac{e^x(x^2 - 2x)}{x^4}\\}
        \onslide<12->{&=& \frac{e^x(x)(x - 2)}{x^4}\\}
        \onslide<13->{&=& \frac{e^x(x - 2)}{x^3}.}
      \end{eqnarray*}
      }
  \end{overprint}
\end{frame}

\begin{frame}{Example}
  Assume
  \begin{multicols}{2}
    \begin{itemize}
    \item<2->
      $f(2) = 1$,
    \item<3->
      $f^\prime(2) = 5$,
    \item<4->
      $g(2) = 3$,
    \item<5->
      $g^\prime(2) = 6$.
    \end{itemize}
  \end{multicols}
  \onslide<6->{
  Let $h(x) = f(x)g(x)$ and $k(x) = f(x)/g(x)$.
  Find
  }
  \begin{multicols}{2}
    \begin{enumerate}[(a)]
    \item<7-|alert@9-12>
      $h^{\prime}(2)$,
    \item<8-|alert@13->
      $k^{\prime}(2)$.
    \end{enumerate}
  \end{multicols}

  \only<9-12>{
    \begin{eqnarray*}
      \onslide<9->{h^\prime(2) &=&} \onslide<10->{f^\prime(2)g(2) +  f(2)g^\prime(2)\\}
      \onslide<11->{&=& 5(3) + 1(6)\\}
      \onslide<12->{&=& 21.}
    \end{eqnarray*}
  }
  \only<13->{
    \begin{eqnarray*}
      \onslide<13->{k^\prime(2) &=&} \onslide<14->{ \frac{f^\prime(2)g(2) - f(2)g^\prime(2)}{g(2)^2}\\}
      \onslide<15->{&=&\frac{5(3) - 1(6)}{3^2}\\}
      \onslide<16->{&=&\frac{15 - 6}{9}} \onslide<17->{=\frac{9}{9}}\onslide<18->{ = 1.}
    \end{eqnarray*}
  }
\end{frame}
\end{document}
