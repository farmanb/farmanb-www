\documentclass[Lecture.tex]{subfiles}
\begin{document}
\section{4.2: Inflection Points}
\begin{frame}{Definition}
  \begin{defn}
    A point at which the graph of a function changes concavity is called an {\it inflection point}.
    \begin{itemize}
    \item<2->
      If $f^\prime$ is differentiable on an interval containing $p$ and  $f^{\prime\prime}(p) = 0$ or $f^{\prime\prime}$ is undefined, then $p$ is a possible inflection point.
    \item<3->
      If the signs of $f^{\prime\prime}(x_1)$ and $f^{\prime\prime}(x_2)$ are different for two points $x_1 < p$ and $p < x_2$, then $p$ is an inflection point.
    \end{itemize}
  \end{defn}
\end{frame}

\begin{frame}{Example}
  Find the inflection points of
  $$f(x) = x^3 - 9x^2 - 48x + 52.$$
  \begin{eqnarray*}
    \onslide<2->{f^\prime(x) &=& 3x^2 - 18x - 48\\}
    \onslide<3->{f^{\prime\prime}(x) &=& 6x - 18 = 6(x - 3)\\}
    \onslide<4->{\Rightarrow f^{\prime\prime}(3) &=& 0\\}
    \onslide<5->{f^{\prime\prime}(0) &=& 6(0 - 3) < 0\\}
    \onslide<6->{f^{\prime\prime}(4) &=& 6(4 - 1) > 0}
  \end{eqnarray*}
  \onslide<7->{Therefore $x = 3$ is an inflection point of $f$.}
\end{frame}

\begin{frame}{Example}
  The point $x = 0$ is a root of the second derivative of $f(x) = x^4$, but it is {\bf not} an inflection point because 
  $$f^{\prime\prime}(x) = 12x^2$$
  never changes sign.
\end{frame}
\end{document}
