\documentclass[Lecture.tex]{subfiles}
\begin{document}
\section{5.5: The Fundamental Theorem of Calculus}

\begin{frame}{The Fundamental Theorem of Calculus}
  Having to estimate definite integrals is incredibly unsatisfying.
  \onslide<2->{Fortunately, we have the following:}
  \onslide<3->{
    \begin{thm}[Fundamental Theorem of Calculus]
      If $F^\prime(t)$ is a continuous function on $[a,b]$, then
      $$\int_a^b F^\prime(t)\dx{t} = F(b) - F(a).$$
    \end{thm}
  }
\end{frame}

\begin{frame}{Example}
  Let $F(t) = \frac{1}{3}x^3$.
  \onslide<2->{
    Differentiating $F$ gives
  }
  $$\onslide<2->{F^\prime(t) =} \onslide<3->{\frac{1}{3}\left(3x^2\right)} \onslide<4->{= x^2}$$
  \onslide<5->{and hence by the Fundamental Theorem of Calculus}
  $$\onslide<6->{\int_0^3 x^2\dx{x} =} \onslide<7->{\int_0^3 F^\prime(x)\dx{x}} \onslide<8->{= F(3) - F(0)} \onslide<9->{= 3^2 - 0^2} \onslide<10->{= 9.}$$
  
  \onslide<11->{
    \begin{rmk}
      Essentially, this says that the area between the derivative of $F$ and the $x$-axis from $a$ and $b$ is just the total change in $F$ on the interval $[a,b]$.
    \end{rmk}
  }
\end{frame}

\begin{frame}{Example}
  For a cost function, $C(q)$, the total change in the cost on $[a,b]$ is given by
  \onslide<2->{
    $$\int_a^b C^\prime(q) \dx{q}.$$
    }

  \onslide<3->{
    Given the marginal cost $C^\prime(q)$ and fixed costs:
    }
  \begin{itemize}
  \item<4->
    Total variable cost to produce $b$ units:
    $$\onslide<5->{C(b) - C(0) = } \onslide<6->{\int_0^b C^\prime(q)\dx{q}.}$$
  \item<7->
    Total cost to produce $b$ units:
    $$\onslide<8->{C(b) = } \onslide<9->{C(b) - C(0) + C(0)} \onslide<10->{= \int_0^b C^\prime(q)\dx{q} + C(0)}.$$
  \end{itemize}
\end{frame}
\end{document}
